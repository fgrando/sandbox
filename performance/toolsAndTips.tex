\documentclass[pdf]{beamer}
\usepackage[utf8]{inputenc}
\usepackage{bookman}
\usetheme{Madrid}

\usepackage{xcolor}
\definecolor{light-gray}{gray}{0.95}
\newcommand{\code}[1]{\colorbox{light-gray}{\texttt{#1}}}
\newcommand{\mono}[1]{\texttt{#1}}

\usepackage{verbatimbox}

\usepackage{color}
\definecolor{lstgrey}{rgb}{0.95,0.95,0.95}
\usepackage{listings}
\lstset{language=C,
       backgroundcolor=\color{lstgrey},
       frame=single,
       basicstyle=\footnotesize\ttfamily,
       captionpos=b,
       tabsize=2,
}


\mode<presentation>{}
%% preamble
\title{Tools and tips}
\subtitle{Notes worth remembering}
\author{Fernando}
%%\email{}
\institute{\href{https://github.com/fgrando}{\emph{github}}}
\date{\today} 

\begin{document}

%% title frame
\begin{frame}
    \titlepage
\end{frame}

\begin{frame}{Topics}
%remember to compile 2 times to build this index
  \tableofcontents
\end{frame}

\section{Motivation}
\begin{frame}{Motivation}
    I create this file to keep track of interesting tools and some tips learned
    along the way.\\
    And also to learn \LaTeX  beamer \mono{:)}
\end{frame}


\section{silversearcher-ag}
\begin{frame}
  \frametitle{silversearcher-ag}
  \code{ag} is like grep or ack. Very useful for searching terms in documentation and code.
  Installation:
  \begin{itemize}
    \item Linux: \mono{apt install silversearcher-ag}
    \item Windows: \url{https://blog.kowalczyk.info/software/the-silver-searcher-for-windows.html}
  \end{itemize}
  \begin{figure}
    \includegraphics[scale=0.4]{data/ag-terminal-linuxpng.png}
    %\caption{console execution}
  \end{figure}
  %Windows port: \href{https://blog.kowalczyk.info/software/the-silver-searcher-for-windows.html}{\beamergotobutton{Link}}
\end{frame}


\section{pdftotext}
\begin{frame}
  \frametitle{pdftotext}
  Extract text from PDF files.\\
  There is also other outputs (\mono{pdftohtml, etc...})\\
  Linux: \mono{apt install poppler-utils}\\
  Windows: \url{https://www.xpdfreader.com/download.html}
\end{frame}


\subsection{convertPdfToTxt.bat}
\begin{frame}[fragile] % fragile is required by verbatim
  \frametitle{\mono{convertPdfToTxt.bat}}
  Script to convert all PDF files in the directory to TXT.
    \begin{lstlisting}
@echo off
for /r %%i in (*.pdf) do (
  echo %%i
  pdftotext "%%i" "%%i.txt"
)
    \end{lstlisting}
\end{frame}






\section{From SVN to GIT}
\begin{frame}
  \frametitle{From SVN to GIT}
  Useful commands:
  \begin{itemize}
    \item \mono{[svn|git] commit -m "first commit"}
    \item \mono{[svn|git] diff <file>}
    
    \item \mono{svn revert <file>}
      \begin{itemize}
      \item \mono{git checkout <file>}
      \end{itemize}
  \end{itemize}
\end{frame}


\section{Assign version numbers to sw}
\begin{frame}
  \frametitle{How to assign version numbers to my sw?}
  Use semver: \url{https://semver.org/}\\
  Given a version number \mono{MAJOR.MINOR.PATCH}, increment the:
  \begin{itemize}
    \item \mono{MAJOR} version when you make incompatible API changes,
    \item \mono{MINOR} version when you add functionality in a backwards compatible manner, and
    \item \mono{PATCH} version when you make backwards compatible bug fixes.
  \end{itemize}
\end{frame}



\section{Message Sequence Chart (\mono{mscgen})}
\begin{frame}[fragile] % fragile is required by verbatim
  \frametitle{Message Sequence Chart (\mono{mscgen})}
  Full featured message chart generator (like graphviz): \url{http://www.mcternan.me.uk/mscgen/}\\
  
  \begin{columns}
    \begin{column}{0.6\textwidth}
      \begin{center}
      \begin{lstlisting}
# Fictional client-server protocol
msc {
 arcgradient = 8;

 a [label="Client"],b [label="Server"];

 a=>b [label="data1"];
 a-xb [label="data2"];
 a=>b [label="data3"];
 a<=b [label="ack1, nack2"];
 a=>b [label="data2", arcskip="1"];
 |||;
 a<=b [label="ack3"];
 |||;
}
      \end{lstlisting}
    \end{center}
    \end{column}
    
    \begin{column}{0.4\textwidth}  %%<--- here
        \begin{center}
          \includegraphics[scale=0.4]{data/mcsgen-example.png}
        \end{center}
    \end{column}
    \end{columns}
\end{frame}














\begin{frame}
  \frametitle{End}
  The end.
\end{frame}

\end{document}
